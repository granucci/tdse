\documentclass[12pt,fleqn]{article}
\usepackage[body={16cm,24cm}]{geometry}
\usepackage[english]{babel}
\usepackage[T1]{fontenc}
\usepackage{amsmath}
\usepackage{enumitem}
\usepackage{scrextend}
\usepackage{verbatim}
\usepackage{alltt}

\setlist{noitemsep}
\setkomafont{labelinglabel}{\normalfont\bfseries}
\setkomafont{labelinglabel}{\tt\bfseries}

\newcommand{\Nomeprog}{\sf QUANT\_EV}
\newcommand{\nomeprog}{\sf quant\_ev}
\newcommand{\hzero}{\rm {\bf H}_{el}^0}
\newcommand{\huno}{\rm {\bf H}_{el}^1}
\newcommand{\Diff}{\Delta\!}
\newenvironment{compactlist}[3]
  {\begin{list}{#1}{\setlength{\leftmargin}{#2em}
                \setlength{\itemsep}{#3ex}
                \setlength{\topsep}{0ex}
                \setlength{\parsep}{0ex}}}
  {\end{list}}
\setlength{\parindent}{0em}
\setlength{\parskip}{2ex}
%\setlength{\topsep}{0ex}
\setlength{\unitlength}{1mm}
%\nohyp
\sloppy
 
\begin{document}
 
\begin{center}
  \large \textbf{PROGRAM} \Nomeprog
\end{center}

Numerical integration of the TDSE for a molecular sistem on a grid. 
The number of dimensions (r) and of electronic states is, in principle, not
limited.

The program reads as input the electronic Hamiltonian in the diabatic basis,
and the time evolution is performed with the Split-Operator method.

Unless otherwise specified, all data are given in atomic units.

%\begin{description}[align=left,labelwidth=3cm]
\begin{labeling}{NAMELIST \&DAT}
  \setlength{\itemsep}{-0.7ex}
  \item[NAMELIST \&DAT]
  \item[NDIM]  number of coordinates.
  \item[RMIN]  minimum values for the coordinates (vector of dimension \texttt{NDIM})
  \item[RMAX]  maximum values for the coordinates (vector of dimension \texttt{NDIM})
  \item[NR]  number of grid points along each coordinate (vector of dimension \texttt{NDIM}). The distance between
    two consecutive point of the grid is dr=(\texttt{RMAX}-\texttt{RMIN})/NR.
  \item[MASSA]  for each coordinate, value of the (reduced) mass (vector of dimension \texttt{NDIM}).
  \item[NSTATI]  number of states.
  \item[ISTATI]  starting state, to which the starting wavepacket is associated. Such a state is in the
    diabatic representation if \texttt{ADIABATIZE} $<2$, otherwise is in the adiabatic representation.
  \item[TIME]  integration time step.
  \item[TCYCLES]  total number of time steps.
  \item[FILE\_WP]  name of the unformatted output file containing the results of the time evolution.
    Default: \texttt{FILE\_WP=WPTOT}.
  \item[WPACK]  \begin{labeling}{gauss}
                       \setlength{\itemsep}{-1ex}
                       \item[gauss]  gaussian starting wavepacket with
                         \begin{align*}
                           <r> &= \mathtt{R0}   \\
                           <p> &= \mathtt{P0}   \\
                           <r^2> - <r>^2 &= \frac{1}{2 * \mathtt{MASSA} * \mathtt{OMEGA}}
                         \end{align*}
                       \item[old]  starting wavepacket read from the end of the file \texttt{FILE\_WP}: a previuous time
                         evolution is continuated. The new results are appended to \texttt{FILE\_WP}.
                       \item[read]  starting wavepacket read from the formatted file \texttt{FILE\_INIWP}.
                     \end{labeling}
  \item[OMEGA] see \texttt{WPACK}.
  \item[R0] see \texttt{WPACK}.
  \item[P0] see \texttt{WPACK}.
  \item[FILE\_MOL] name of the input file containing energies and couplings, in the diabatic representation.
    See the ``quant\_sys'' module.
  \item[FILE\_INIWP] name of the formatted input file containing the starting wavepacket if \texttt{WPACK='read'}. 
    \texttt{FILE\_INIWP} contains as many record as the number of grid points. Each
    record has the form \\
    $r$ ~~~ $wpx$ \\
    where $r$(1:\texttt{NDIM}) are the coordinate values in the given point of the grid and $wpx$ is the value of the
    wavepacket in that point for the state \texttt{ISTATI}. It is assumed that the starting wavepacket is real.
    Only if \texttt{WPACK='read'}.
  \item[HCOMPLEX] \texttt{F}, the diabatic Hamiltonian is real (default) ; \\
    \texttt{T}, the diabatic Hamiltonian is complex.
  \item[ADIABATIZE] \begin{labeling}{3}
                       \setlength{\itemsep}{-0.5ex}
                       \item[0] initial conditions and results in the diabatic representation (default).
                       \item[1] initial conditions in the diabatic representation, results in the adiabatic
                      representation. 
                       \item[2] initial conditions and results in the adiabatic representation.
                       \item[3] initial conditions in the adiabatic representation, results in the diabatic
                      representation.
                    \end{labeling}
  \item[NPRT] number of time steps between two consecutive writings. Default, \texttt{NPRT=1}.
  \item[NPRT\_WP] like \texttt{NPRT}, but for the wavefunction. Default, \texttt{NPRT\_WP=NPRT}.
  \item[ENE\_ADD] factor to be added to the total energy. Default, \texttt{ENE\_ADD=0}.
  \item[IWRT] \begin{labeling}{>9}
                       \setlength{\itemsep}{-0.5ex}
                       \item[0] normal writing (default).
                       \item[>0] total and kinetic energies are written.
                       \item[>1] the wavepacket is written on \texttt{FILE\_WP}. If \texttt{NDIM=1} the
                         wavepackets are also written on the formatted file 'WP'.
                       \item[>2] the grid, together with the adiabatic energies, is written.
                       \item[>3] diabatic populations, diabatic energies, and diabatic epot matrix 
                         (irrespective of the value of \texttt{ADIABATIZE}). 
                       \item[>9] the grid is written and the calculation is stopped.
                    \end{labeling}
  \item[THRWP] threshold for writings on the formatted file 'WP'. A point of the grid is
    written only if a) the population in that point of one of the wavepackets is larger than 
    the threshold, b) is found between two points with population larger than the threshold.
  \item[THR\_ABSPOT] threshold for the absorbing potential. If the absolute value of the absorbing potential
    (imaginary part of the diagonal terms in the diabatic Hamiltonian) is larger than the threshold, the
    absorbing potential is applied. Relevant only for \texttt{HCOMPLEX=T}. Default,
    \texttt{THR\_ABSPOT=1.d-7}.
  \item[RADIATION] \texttt{F}, normal calculation (default); \\ 
    \texttt{T}, interaction with the electromagnetic radiation.
\end{labeling}
%\end{description}


\end{document}
